\documentclass{article}

\usepackage{cmap}
\usepackage[T2A]{fontenc}
\usepackage[utf8]{inputenc}
\usepackage[russian]{babel}
\usepackage{amsmath,amssymb,amsthm}
\usepackage[pdftex,colorlinks=true,linkcolor=blue,urlcolor=red,unicode=true,hyperfootnotes=false,bookmarksnumbered]{hyperref}
\usepackage{indentfirst}
\usepackage{stmaryrd}
\usepackage{listings}
\usepackage{color}

\newcommand{\E}{\ensuremath{\mathsf{E}}}
\newcommand{\D}{\ensuremath{\mathsf{D}}}
\newcommand{\Prb}{\ensuremath{\mathsf{P}}}

\newcommand{\eps}{\varepsilon}
\renewcommand{\phi}{\varphi}

\renewcommand{\le}{\leqslant}
\renewcommand{\leq}{\leqslant}
\renewcommand{\ge}{\geqslant}
\renewcommand{\geq}{\geqslant}

\begin{document}

    \title{Лабараторная работа № 3\\ 3 курс, 3 группа}
    \author{Владислав Адаменко}
    \maketitle

    \section{Цель работы}
    Целью данной лабораторной работы является закрепление теоретических знаний по растеризации отрезков и кривых с использованием базовых алгоритмов, таких как пошаговый алгоритм, алгоритм Брезенхема.

    \section{Задачи работы}
    - Реализовать пошаговый алгоритм растеризации отрезков.

    - Реализовать алгоритм Брезенхема.

    - Построить дружелюбный и удобный графический интерфейс, включающий масштаб, систему координат, оси, линии сетки и подписи.

    \section{Использование средств разработки}
    Для выполнения лабораторной работы использовались язык программирования Python и библиотека Tkinter для создания графического интерфейса.

    \section{Ход работы}
    В процессе выполнения лабораторной работы были реализованы алгоритмы растеризации отрезков и окружностей. Для проверки работоспособности алгоритмов был создан графический интерфейс с использованием библиотеки Tkinter. Примеры вызова алгоритмов были добавлены в код для отображения результатов.

    \section{Выводы}
    В ходе выполнения лабораторной работы были закреплены теоретические знания по растеризации геометрических объектов. Реализованные алгоритмы успешно работают, что подтверждено графическим интерфейсом.

    \subsection{Временные характеристики реализованных алгоритмов}
    В ходе выполнения лабораторной работы были закреплены теоретические знания по растеризации геометрических объектов. Реализованные алгоритмы обладают следующей асимптотикой:

    - Алгоритм пошагового рисования отрезков: \(O(n)\), где \(n\) - количество пикселей на отрезке.

    - Алгоритм Брезенхема для отрезков: \(O(\Delta x)\), где \(\Delta x\) - разница по x между конечной и начальной точкой отрезка.

    - Алгоритм Брезенхема для окружности: \(O(r)\), где \(r\) - радиус окружности.

    Реализованные алгоритмы работают эффективно и обладают различной асимптотикой в зависимости от характеристик геометрических объектов.


\end{document}
