\documentclass{article}

\usepackage{cmap}
\usepackage[T2A]{fontenc}
\usepackage[utf8]{inputenc}
\usepackage[russian]{babel}
\usepackage{amsmath,amssymb,amsthm}
\usepackage[pdftex,colorlinks=true,linkcolor=blue,urlcolor=red,unicode=true,hyperfootnotes=false,bookmarksnumbered]{hyperref}
\usepackage{indentfirst}
\usepackage{stmaryrd}
\usepackage{listings}
\usepackage{color}

\newcommand{\E}{\ensuremath{\mathsf{E}}}
\newcommand{\D}{\ensuremath{\mathsf{D}}}
\newcommand{\Prb}{\ensuremath{\mathsf{P}}}

\newcommand{\eps}{\varepsilon}
\renewcommand{\phi}{\varphi}

\renewcommand{\le}{\leqslant}
\renewcommand{\leq}{\leqslant}
\renewcommand{\ge}{\geqslant}
\renewcommand{\geq}{\geqslant}

\begin{document}

    \title{Отчет о лабораторной работе №1\\ Цветовые модели}
    \author{Владислав Адаменко}
    \maketitle

    \section{Цель работы}
    Целью данной лабораторной работы является изучение цветовых моделей, таких как RGB, CMYK, HSV, HLS, и их преобразований. Также необходимо создать приложение или веб-приложение, которое позволяет пользователю выбирать и изменять цвета, отображая их составляющие в трех моделях одновременно.

    \section{Задачи работы}
    В рамках лабораторной работы были поставлены следующие задачи:
    \begin{itemize}
        \item Изучить цветовые модели: RGB, CMYK, HSV, HLS и их компоненты.
        \item Разработать приложение/веб-приложение, предоставляющее пользователю возможность выбирать цвета и изменять их.
        \item Обеспечить интерактивность в изменении цветов.
        \item Автоматически пересчитывать составляющие цвета в двух других цветовых моделях при изменении любой компоненты цвета.
    \end{itemize}

    \section{Использование средств разработки}
    Для выполнения задачи был использован язык программирования Python с библиотеками PyQt5 для создания графического интерфейса.

    \section{Ход работы}

    В процессе выполнения лабораторной работы были выполнены следующие основные шаги:

    \begin{enumerate}
        \item В первую очередь был создан пользовательский интерфейс для приложения с использованием Qt 5 Designer.

        \item После создания интерфейса в Qt 5 Designer, файл .ui был сконвертирован в .py файл с помощью утилиты pyuic5.

        \item Написание основной логики программы.
        Были описаны методы для обработки сигналов, связанных с изменением цветовых компонент.
        При изменении любой компоненты цвета, программа автоматически пересчитывала значения в других цветовых моделях и обновляла интерфейс.
        Таким образом, обеспечивалась интерактивность в изменении цветов.

    \end{enumerate}

    \section{Выводы}
    В результате выполнения лабораторной работы было создано приложение для работы с цветовыми моделями, позволяющее выбирать и изменять цвета в разных моделях, а также автоматически пересчитывать значения в других моделях при изменении компонент цвета.
\end{document}
