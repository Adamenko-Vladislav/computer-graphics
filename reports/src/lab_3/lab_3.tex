\documentclass{article}

\usepackage{cmap}
\usepackage[T2A]{fontenc}
\usepackage[utf8]{inputenc}
\usepackage[russian]{babel}
\usepackage{amsmath,amssymb,amsthm}
\usepackage[pdftex,colorlinks=true,linkcolor=blue,urlcolor=red,unicode=true,hyperfootnotes=false,bookmarksnumbered]{hyperref}
\usepackage{indentfirst}
\usepackage{stmaryrd}
\usepackage{listings}
\usepackage{color}

\newcommand{\E}{\ensuremath{\mathsf{E}}}
\newcommand{\D}{\ensuremath{\mathsf{D}}}
\newcommand{\Prb}{\ensuremath{\mathsf{P}}}

\newcommand{\eps}{\varepsilon}
\renewcommand{\phi}{\varphi}

\renewcommand{\le}{\leqslant}
\renewcommand{\leq}{\leqslant}
\renewcommand{\ge}{\geqslant}
\renewcommand{\geq}{\geqslant}

\begin{document}

    \title{Лабараторная работа № 3\\ 3 курс, 3 группа}
    \author{Владислав Адаменко}
    \maketitle
    
    \section{Цель работы}
        Разработать приложение для обработки изображений с использованием глобальной пороговой обработки и адаптивной пороговой обработки.
        Реализовать также поэлементные операции и линейное контрастирование.
        Приложение должно иметь графический интерфейс и предоставлять возможность загрузки изображений для обработки.
        В качестве тестовых данных необходимо подготовить базу изображений, включающую в себя зашумленные, размытые, и малоконтрастные изображения.

    \section{Задачи работы}
        Задача включает в себя разработку, тестирование и документацию приложения.

    \section{Использование средств разработки}
    \begin{itemize}
        \item \textbf{OpenCV (Open Source Computer Vision Library):} Библиотека OpenCV была использована для загрузки изображений, их обработки и сохранения обработанных изображений.

        \item \textbf{Tkinter:} Tkinter - это стандартная библиотека Python для создания графического интерфейса пользователя (GUI). В данном случае, Tkinter использовался для создания простого GUI, позволяющего пользователю выбирать входную и выходную директории, а также запускать обработку изображений.

        \item \textbf{Методы обработки изображений:} Для выполнения задачи обработки изображений были разработаны и использованы методы, такие как пороговая обработка методами Бернсена и Ниблека, адаптивная пороговая обработка, логарифмическое преобразование, линейное контрастирование.
    \end{itemize}
    \section{Ход работы}

    В рамках выполнения лабораторной работы были выполнены следующие шаги:

    \begin{enumerate}
        \item \textbf{Написание кода:} В первую очередь был написан код, реализующий методы обработки изображений. Для этого были использованы библиотеки OpenCV и NumPy. Код включает в себя реализацию методов пороговой обработки, логарифмического преобразования, линейного контрастирования.

        \item \textbf{Подборка изображений для тестирования методов обработки изображений:} Для тестирования разработанных методов обработки изображений была проведена подборка изображений различных типов. В эту подборку включены изображения с разными характеристиками, такими как зашумление, размытие, малоконтрастные изображения и другие типы. Эти изображения используются для проверки корректности работы методов обработки.

        \item \textbf{Тестирование приложения:} Для проверки работы разработанных методов было проведено тестирование приложения. Каждый метод был применен к выбранным изображениям, и результаты обработки были сохранены. Затем была проведена оценка качества обработки, а также изучение влияния различных параметров и настроек методов на результаты обработки.

    \end{enumerate}

    В результате выполнения этих шагов были получены обработанные изображения и проведена проверка корректности работы методов обработки изображений.


    \section{Выводы}

    В ходе выполнения данной лабораторной работы были реализованы и протестированы различные методы обработки изображений. Были разработаны методы пороговой обработки, включая методы Бернсена и Ниблека, адаптивное пороговое преобразование, логарифмическое преобразование и линейное контрастирование.

    Основные результаты работы:

    \begin{itemize}
        \item Был разработан и реализован код, позволяющий применять различные методы обработки изображений к входным изображениям.

        \item Проведена подборка изображений для тестирования методов обработки, включая изображения с разными характеристиками.

        \item Проведено тестирование разработанных методов с использованием выбранных изображений. Результаты обработки были сохранены и проанализированы.

        \item Оценено влияние различных параметров и настроек методов на результаты обработки.

    \end{itemize}

    В результате работы было показано, что разработанные методы обработки изображений способны корректно обрабатывать разнообразные типы изображений и могут быть эффективно использованы для улучшения качества изображений. Лабораторная работа позволила закрепить знания о методах обработки изображений и применить их на практике.


\end{document}
